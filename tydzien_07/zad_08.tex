\subsection{Zadanie 8.}

Długość łodygi pewnego gatunku roślin ma rozkład normalny o parametrach $\mu = 70$ cm oraz $\sigma^2 = 27.04$ $\mbox{cm}^2$.
Oblicz prawdopodobieństwo zdarzenia, że wylosowana roślina ma łodygę o długości:

\begin{enumerate}[label=(\alph*)]
\item co najwyżej 68 cm
\item co najmniej  72 cm
\item co najwyżej -10 cm
\end{enumerate}


Rozwiązanie:

$$ X \sim N ( \mu, \sigma ) $$
$$\mu = 70$$
$$ \sqrt{ \sigma^2 } = \sqrt{27.04} = 5.2 $$
$$ X \sim N ( 70, 5.2 ) $$
$$ z = \frac{X - 70}{5.2} $$ \\

\begin{enumerate}

\item
$$ P( X \le 68 ) = P(  \frac{X - 70}{5.2} \le \frac{68 - 70}{5.2} ) = $$
$$ = P( z \le \frac{-2}{5.2} ) =  P( z \le -0.38 ) = \Phi(-0.38) = $$
$$ = 1 - \Phi(0.38) = 1 - 0.64803 = 0.35197 $$ \\

\item
$$ P( X \ge 72 ) = P( \frac{X - 70}{5.2} \ge \frac{72 - 70}{5.2} ) = P( z \ge \frac{2}{5.2} ) =  $$
$$ = P(z \ge 0.38 ) = 1 - P( z \le 0.38 ) = 1 - 0.64803 = 0.35197 $$ \\

\item
$$ P( X \le -10 ) = P( \frac{X - 70}{5.2} \le \frac{-10 - 70}{5.2} ) = $$
$$ = P( z \le \frac{-80}{5.2} ) = P( z \le -15.38 ) = . . .$$
\end{enumerate}
